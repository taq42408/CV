%!TEX TS-program = xelatex
%!TEX encoding = UTF-8 Unicode
% Awesome CV LaTeX Template for CV/Resume
%
% This template has been downloaded from:
% https://github.com/posquit0/Awesome-CV
%
% Author:
% Claud D. Park <posquit0.bj@gmail.com>
% http://www.posquit0.com
%
%
% Adapted to be an Rmarkdown template by Mitchell O'Hara-Wild
% 23 November 2018
%
% Template license:
% CC BY-SA 4.0 (https://creativecommons.org/licenses/by-sa/4.0/)
%
%-------------------------------------------------------------------------------
% CONFIGURATIONS
%-------------------------------------------------------------------------------
% A4 paper size by default, use 'letterpaper' for US letter
\documentclass[11pt,a4paper,]{awesome-cv}

% Configure page margins with geometry
\usepackage{geometry}
\geometry{left=1.4cm, top=.8cm, right=1.4cm, bottom=1.8cm, footskip=.5cm}


% Specify the location of the included fonts
\fontdir[fonts/]

% Color for highlights
% Awesome Colors: awesome-emerald, awesome-skyblue, awesome-red, awesome-pink, awesome-orange
%                 awesome-nephritis, awesome-concrete, awesome-darknight

\colorlet{awesome}{awesome-red}

% Colors for text
% Uncomment if you would like to specify your own color
% \definecolor{darktext}{HTML}{414141}
% \definecolor{text}{HTML}{333333}
% \definecolor{graytext}{HTML}{5D5D5D}
% \definecolor{lighttext}{HTML}{999999}

% Set false if you don't want to highlight section with awesome color
\setbool{acvSectionColorHighlight}{true}

% If you would like to change the social information separator from a pipe (|) to something else
\renewcommand{\acvHeaderSocialSep}{\quad\textbar\quad}

\def\endfirstpage{\newpage}

%-------------------------------------------------------------------------------
%	PERSONAL INFORMATION
%	Comment any of the lines below if they are not required
%-------------------------------------------------------------------------------
% Available options: circle|rectangle,edge/noedge,left/right

\name{Tomas}{Quezada}

\position{Entomology Technician}
\address{Cornell University}

\mobile{+1 956-286-1840}
\email{\href{mailto:TAQ3@cornell.edu}{\nolinkurl{TAQ3@cornell.edu}}}
\github{\url{https://github.com/taq42408}}

% \gitlab{gitlab-id}
% \stackoverflow{SO-id}{SO-name}
% \skype{skype-id}
% \reddit{reddit-id}


\usepackage{booktabs}

\providecommand{\tightlist}{%
	\setlength{\itemsep}{0pt}\setlength{\parskip}{0pt}}

%------------------------------------------------------------------------------

\usepackage{color}
\usepackage{fancyvrb}
\newcommand{\VerbBar}{|}
\newcommand{\VERB}{\Verb[commandchars=\\\{\}]}
\DefineVerbatimEnvironment{Highlighting}{Verbatim}{commandchars=\\\{\}}
% Add ',fontsize=\small' for more characters per line
\usepackage{framed}
\definecolor{shadecolor}{RGB}{248,248,248}
\newenvironment{Shaded}{\begin{snugshade}}{\end{snugshade}}
\newcommand{\AlertTok}[1]{\textcolor[rgb]{0.94,0.16,0.16}{#1}}
\newcommand{\AnnotationTok}[1]{\textcolor[rgb]{0.56,0.35,0.01}{\textbf{\textit{#1}}}}
\newcommand{\AttributeTok}[1]{\textcolor[rgb]{0.77,0.63,0.00}{#1}}
\newcommand{\BaseNTok}[1]{\textcolor[rgb]{0.00,0.00,0.81}{#1}}
\newcommand{\BuiltInTok}[1]{#1}
\newcommand{\CharTok}[1]{\textcolor[rgb]{0.31,0.60,0.02}{#1}}
\newcommand{\CommentTok}[1]{\textcolor[rgb]{0.56,0.35,0.01}{\textit{#1}}}
\newcommand{\CommentVarTok}[1]{\textcolor[rgb]{0.56,0.35,0.01}{\textbf{\textit{#1}}}}
\newcommand{\ConstantTok}[1]{\textcolor[rgb]{0.00,0.00,0.00}{#1}}
\newcommand{\ControlFlowTok}[1]{\textcolor[rgb]{0.13,0.29,0.53}{\textbf{#1}}}
\newcommand{\DataTypeTok}[1]{\textcolor[rgb]{0.13,0.29,0.53}{#1}}
\newcommand{\DecValTok}[1]{\textcolor[rgb]{0.00,0.00,0.81}{#1}}
\newcommand{\DocumentationTok}[1]{\textcolor[rgb]{0.56,0.35,0.01}{\textbf{\textit{#1}}}}
\newcommand{\ErrorTok}[1]{\textcolor[rgb]{0.64,0.00,0.00}{\textbf{#1}}}
\newcommand{\ExtensionTok}[1]{#1}
\newcommand{\FloatTok}[1]{\textcolor[rgb]{0.00,0.00,0.81}{#1}}
\newcommand{\FunctionTok}[1]{\textcolor[rgb]{0.00,0.00,0.00}{#1}}
\newcommand{\ImportTok}[1]{#1}
\newcommand{\InformationTok}[1]{\textcolor[rgb]{0.56,0.35,0.01}{\textbf{\textit{#1}}}}
\newcommand{\KeywordTok}[1]{\textcolor[rgb]{0.13,0.29,0.53}{\textbf{#1}}}
\newcommand{\NormalTok}[1]{#1}
\newcommand{\OperatorTok}[1]{\textcolor[rgb]{0.81,0.36,0.00}{\textbf{#1}}}
\newcommand{\OtherTok}[1]{\textcolor[rgb]{0.56,0.35,0.01}{#1}}
\newcommand{\PreprocessorTok}[1]{\textcolor[rgb]{0.56,0.35,0.01}{\textit{#1}}}
\newcommand{\RegionMarkerTok}[1]{#1}
\newcommand{\SpecialCharTok}[1]{\textcolor[rgb]{0.00,0.00,0.00}{#1}}
\newcommand{\SpecialStringTok}[1]{\textcolor[rgb]{0.31,0.60,0.02}{#1}}
\newcommand{\StringTok}[1]{\textcolor[rgb]{0.31,0.60,0.02}{#1}}
\newcommand{\VariableTok}[1]{\textcolor[rgb]{0.00,0.00,0.00}{#1}}
\newcommand{\VerbatimStringTok}[1]{\textcolor[rgb]{0.31,0.60,0.02}{#1}}
\newcommand{\WarningTok}[1]{\textcolor[rgb]{0.56,0.35,0.01}{\textbf{\textit{#1}}}}


% Pandoc CSL macros
\newlength{\cslhangindent}
\setlength{\cslhangindent}{1.5em}
\newlength{\csllabelwidth}
\setlength{\csllabelwidth}{2em}
\newenvironment{CSLReferences}[3] % #1 hanging-ident, #2 entry spacing
 {% don't indent paragraphs
  \setlength{\parindent}{0pt}
  % turn on hanging indent if param 1 is 1
  \ifodd #1 \everypar{\setlength{\hangindent}{\cslhangindent}}\ignorespaces\fi
  % set entry spacing
  \ifnum #2 > 0
  \setlength{\parskip}{#2\baselineskip}
  \fi
 }%
 {}
\usepackage{calc}
\newcommand{\CSLBlock}[1]{#1\hfill\break}
\newcommand{\CSLLeftMargin}[1]{\parbox[t]{\csllabelwidth}{\honortitlestyle{#1}}}
\newcommand{\CSLRightInline}[1]{\parbox[t]{\linewidth - \csllabelwidth}{\honordatestyle{#1}}}
\newcommand{\CSLIndent}[1]{\hspace{\cslhangindent}#1}

\begin{document}

% Print the header with above personal informations
% Give optional argument to change alignment(C: center, L: left, R: right)
\makecvheader

% Print the footer with 3 arguments(<left>, <center>, <right>)
% Leave any of these blank if they are not needed
% 2019-02-14 Chris Umphlett - add flexibility to the document name in footer, rather than have it be static Curriculum Vitae


%-------------------------------------------------------------------------------
%	CV/RESUME CONTENT
%	Each section is imported separately, open each file in turn to modify content
%------------------------------------------------------------------------------



\begin{Shaded}
\begin{Highlighting}[]
\NormalTok{readExcelSheets }\OtherTok{\textless{}{-}} \ControlFlowTok{function}\NormalTok{(filename)\{}
  \CommentTok{\# read all sheets in .xlsx }
\NormalTok{  all }\OtherTok{\textless{}{-}}\NormalTok{ readxl}\SpecialCharTok{::}\FunctionTok{excel\_sheets}\NormalTok{(filename)}
  
  \CommentTok{\# import each sheet into a list using readxl::read\_excel}
\NormalTok{  list }\OtherTok{\textless{}{-}} \FunctionTok{lapply}\NormalTok{(all, }\ControlFlowTok{function}\NormalTok{(x) readxl}\SpecialCharTok{::}\FunctionTok{read\_excel}\NormalTok{(filename, }\AttributeTok{sheet =}\NormalTok{ x))}
  
  \CommentTok{\# save sheet name for each sheet (list)}
  \FunctionTok{names}\NormalTok{(list) }\OtherTok{\textless{}{-}}\NormalTok{ all}
  
  \CommentTok{\# breaks up list and creates a dataframe for each sheet with df names matching sheet names}
  \FunctionTok{list2env}\NormalTok{(list, }\AttributeTok{envir =}\NormalTok{ .GlobalEnv)}
\NormalTok{\}}
\end{Highlighting}
\end{Shaded}

\hypertarget{some-stuff-about-me}{%
\section{Some stuff about me}\label{some-stuff-about-me}}

\begin{itemize}
\tightlist
\item
  I poisoned myself doing research.
\item
  I was the first woman to win a Nobel prize
\item
  I was the first person and only woman to win a Nobel prize in two
  different sciences.
\end{itemize}

\hypertarget{education}{%
\section{Education}\label{education}}

\begin{cventries}
    \cventry{Informal studies}{Flying University}{Warsaw, Poland}{1889-91}{}\vspace{-4.0mm}
    \cventry{Master of Physics}{Sorbonne Université}{Paris, France}{1893}{}\vspace{-4.0mm}
    \cventry{Master of Mathematics}{Sorbonne Université}{Paris, France}{1894}{}\vspace{-4.0mm}
\end{cventries}

\hypertarget{nobel-prizes}{%
\section{Nobel Prizes}\label{nobel-prizes}}

\begin{cvhonors}
    \cvhonor{}{Nobel Prize in Physics}{Awarded for her work on radioactivity with Pierre Curie and Henri Becquerel}{1903}
    \cvhonor{}{Nobel Prize in Chemistry}{Awarded for the discovery of radium and polonium}{1911}
\end{cvhonors}

\hypertarget{publications}{%
\section{Publications}\label{publications}}

\hypertarget{bibliography}{}
\leavevmode\vadjust pre{\hypertarget{ref-R-vitae}{}}%
\CSLLeftMargin{1. }%
\CSLRightInline{O'Hara-Wild, M., \& Hyndman, R. (2022). \emph{Vitae:
Curriculum vitae for r markdown}.
\url{https://CRAN.R-project.org/package=vitae}}

\leavevmode\vadjust pre{\hypertarget{ref-R-tibble}{}}%
\CSLLeftMargin{2. }%
\CSLRightInline{Müller, K., \& Wickham, H. (2022). \emph{Tibble: Simple
data frames}. \url{https://CRAN.R-project.org/package=tibble}}


\label{LastPage}~
\end{document}
